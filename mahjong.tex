\documentclass{jsarticle}
\usepackage{multicol}

% sourcecode highlight
\usepackage{color}
\usepackage{listings,jlisting}

%jlisting (sourcecode highlight)
\lstset{
language={C},
backgroundcolor={\color[gray]{.85}},
basicstyle={\small},
identifierstyle={\small},
commentstyle={\small\ttfamily \color[rgb]{0,0.5,0}},
keywordstyle={\small \color[rgb]{0,0,0}},
ndkeywordstyle={\small},
stringstyle={\small\ttfamily \color[rgb]{0,0,1}},
frame={tb},
breaklines=true,
columns=[l]{fullflexible},
numbers=left,
xrightmargin=0zw,
xleftmargin=3zw,
numberstyle={\scriptsize},
stepnumber=1,
numbersep=1zw,
morecomment=[l]{//}
}

\title{麻雀点数計算プログラム\ 注釈書}
\author{14EC004\ 飯田頌平}

\begin{document}
\maketitle

\begin{multicols}{2}

\section{はじめに}
本稿は吉田さんの麻雀点数計算プログラム\cite{yoshida}
の注釈書にあたる。

このプログラムはカメラで麻雀の手牌を撮影すると
点数を自動で計算するというものであり、
大まかにサーバ、画像認識、点数計算の三つのモジュールに分かれている。

このうち本稿では画像認識の部分について触れる。
画像認識の実装ファイルTemplateMatching.scalaを付録に掲載するので、参考にしながら読むこと。

\section{画像認識のフロー}

画像認識はパターンマッチングによって行われている。
よって、最初に訓練データからテンプレートを生成し、その後テストデータとテンプレートを照合させて結果を求める。
まずはテンプレートの生成手順を示す。
\begin{enumerate}
\item 雀牌群と背景を判別し、雀牌群だけを切り取る
\item 雀牌群を黒と白で二値化する
\item 雀牌群から雀牌ひとつあたりの縦幅と横幅を求める
\item 白と判別された誤差(牌の隅)を黒く塗りつぶす
\end{enumerate}

%\begin{lstlisting}[caption=RaspiAudio/voice/lowpassFilter.py,label=filter]
%
%\end{lstlisting}

\begin{thebibliography}{9}
\bibitem{yoshida}
	Sanshiro Yoshida.mahjongs.\\
	https://github.com/halcat0x15a/mahjongs
\end{thebibliography}

\end{multicols}
\end{document}