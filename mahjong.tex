\documentclass{jsarticle}
\usepackage{multicol}

% sourcecode highlight
\usepackage{color}
\usepackage{listings,jlisting}

%jlisting (sourcecode highlight)
\lstset{
language={C},
backgroundcolor={\color[gray]{.85}},
basicstyle={\small},
identifierstyle={\small},
commentstyle={\small\ttfamily \color[rgb]{0,0.5,0}},
keywordstyle={\small \color[rgb]{0,0,0}},
ndkeywordstyle={\small},
stringstyle={\small\ttfamily \color[rgb]{0,0,1}},
frame={tb},
breaklines=true,
columns=[l]{fullflexible},
numbers=left,
xrightmargin=0zw,
xleftmargin=3zw,
numberstyle={\scriptsize},
stepnumber=1,
numbersep=1zw,
morecomment=[l]{//}
}

%comment out multiple line
\usepackage{comment}


\title{麻雀点数計算プログラム\ 注釈書}
\author{14EC004\ 飯田頌平}

\begin{document}
\maketitle

\begin{multicols}{2}

\section{はじめに}
本稿は吉田さんの麻雀点数計算プログラム\cite{yoshida}
の注釈書にあたる。

このプログラムはカメラで麻雀の手牌を撮影すると
点数を自動で計算するというものであり、
大まかにサーバ、画像認識、点数計算の三つのモジュールに分かれている。

このうち本稿では画像認識の部分について触れる。
画像認識の実装ファイルTemplateMatching.scala
およびユーザ定義関数の実装ファイルpackage.scala
を付録に掲載するので、参考にしながら読むこと。

\section{画像認識のフロー}

画像認識はパターンマッチングによって行われている。
よって、最初に訓練データからテンプレートを生成し、その後テストデータとテンプレートを照合させて結果を求める。
まずはテンプレートの生成手順を示す。
\begin{enumerate}
\item 雀牌と背景を判別し、雀牌だけを切り取る
\item 雀牌を黒と白で二値化する
\item 雀牌から牌ひとつあたりの縦幅と横幅を求める
\item 白と判別された誤差(牌の隅)を黒く塗りつぶす
\item 雀牌から牌のシーケンスを抽出する
\item 雀牌の絵柄部分の輪郭を得る
\item 雀牌のグレースケールからテンプレートを生成する
\end{enumerate}

詳細な説明については後述するとして、
次に照合の手順を示す。
\begin{enumerate}
\item 手牌と背景を区別し、手牌だけを切り取る
\item テンプレートから各牌を回転させたイメージを得る
\item 手牌とテンプレート(及び回転イメージ)を照合させ、座標の形で一致箇所を得る
\item 座標を基に手牌を判定する
\item 門前か否かを判定する
\end{enumerate}

こうして手牌を認識することができた。
次章以降ではこれらの流れの詳細について述べる。

\section{テンプレートの生成}

テンプレートの生成にあたって、
まずは元画像を用意する。
以下にその要件を示す。
\begin{itemize}
\item 雀牌を4*9の矩形上に隙間なく並べる
\item 牌の位置は固定である \\
		  一萬からはじまるSEQが予め定められている
\item 背景は一色のものを選ぶ
\item 雀牌と背景以外のものを除外する
\item 背景もできるだけ写り込まないようにする
\item 撮影時に、各牌が等しい高さと幅を持つよう注意する
\end{itemize}

背景には麻雀マットなどを用いれば良い。
また斜めから撮影された画像からは正しいテンプレートが得られないことに気を付ける。

元画像はプログラムへMat型で渡される。
Mat型はopencv\footnote{https://github.com/bytedeco/javacv}のライブラリが提供する行列の型であり、
opencvではこの型で画像データを扱う。
createTemplateの引数matが元画像のデータを示す変数である。

matに対して最初に行われる処理は、定数MaxResolutionのサイズに近づくよう
resizeで画像サイズを修正する。
大きい画像であると処理時間が増えてしまうため、これによって効率化する。

次に雀牌と背景を識別し、雀牌だけを切り取るcropを実行する。
cropは元画像を引数に取り、Buffer型のシーケンスで識別結果の画像を返す。
(なおBufferはタプルであり、画像だけでなく輪郭の二次元配列も返しているが、
この値はcreateTemplateでは参照しない。)

cropの実装は、
まずfindContoursによって元画像を輪郭で分離し、
minAreaRectで矩形を得て、
warpAffineで矩形の傾きを底面と垂直になるよう修正し、
getRectSubPixで矩形領域のピクセル値を得ることで、
入力画像を輪郭毎に分離した上で補正をかけた画像を得るようになっている。

白と黒で二値化されたテンプレート画像は、
牌の集合の縁は白、背景は黒と別の色で分けられているため、
findCountoursによって輪郭を得られるようになっている。
得た輪郭countourのメソッドtoArrayを用いれば、
minAreaRectで矩形を引くべき座標を得られるのである。

こうして得られた分離後の画像のシーケンス(返り値はBuffer型であることに注意)から
headOptionを使うことで先頭要素を取り出せる。
findContoursでは大きさ順に画像をソートする作用があるので、
先頭要素には一番大きな画像が入る。
一番大きな画像が雀牌であるという前提であるなら(元画像の背景部分が大きすぎるとこの前提は崩れる)、
この先頭要素は雀牌であるため、雀牌のテンプレートは引数templateに束縛される。

thresholdによってテンプレートを黒と白で二値化し、
widthとheightに牌ひとつあたりの幅と高さを代入する。
このとき雀牌の並びが4*9でなかったとき、想定外の値を取ってしまうので注意すること。

こうして得られた二値化後の画像について見てみると、
牌の模様が白、牌の背景が黒で表されていることが分かる。
しかしそれと同時に、牌と牌の間なども白く判定されてしまっている。
牌の模様以外が白と認識されてはパターンマッチに失敗するため、
次はこの誤判定部分を黒く塗りつぶす。
それにはfloodFillを使う。
floodFillは座標を指定し、連結部分を指定した色で塗りつぶす関数である。
これを牌と牌の間にすべてに対して実行することで、
誤判定が塗りつぶされる。

そして牌の模様だけを抽出する。
findContoursによって再び輪郭ごとの分離をかければ、
模様だけを抽出することができる。
四筒のように模様が複数に分離されてしまうケースの場合は、
convexHullによって結合する。
最後に、得られた模様の中で、もっとも面積の大きなものをsizeに代入する。
マッチングを行う際にはこのサイズを基準にするためである。

ここまでで二値化の済んだ雀牌のテンプレート画像が得られた。
今度はこれを牌ひとつひとつに分離して、牌ごとのテンプレートを得る。

それにはgridを用いる。
gridはsubmatを実装に含み、元画像から矩形範囲を抽出できる関数である。
gridで牌を4*9に区切り、zipで牌ごとに模様の輪郭の情報を付与し、
それらにflatMapを挟んでgetRectSubPixをかけることによって、
輪郭だけが抽出されたすべての牌のテンプレートを得ることができる。
なお、このテンプレートは白黒ではなくグレースケールである点に注意。

以上でテンプレートの生成が完了した。

\section{テストデータとの照合}

照合は関数recognizeによって実装されている。
この引数matに手牌の画像データを、templatesにテンプレートの画像データを渡す。

手牌のデータはcropによって背景から切り離される。
しかしテンプレートをcropで切り離したときと違い、
この場合は牌のデータが複数に分かれる。
手牌には門前と鳴きがあり、鳴いた牌は手牌と離して置くためだ。

手牌のデータはcollectによってhandに束縛される。
テンプレートに向きを回転させた牌を加えてtilesに代入すると、
いよいよgoでパターンマッチを行う。

%疑問提起
\begin{comment}
collectはfilterとmapを合わせたような関数
条件を満たした結果のみのシーケンスを作る
条件文はcase(hand,contour) if hand.size.area>0
だがこれでは何の条件判定をしているのか?
\end{comment}

goの内部では、まず各テンプレートに対してOpenCVのテンプレートマッチング関数たるmatchTemplateを呼ぶ。
その結果がresultへと代入されるので、
minMaxLocを通して最もスコアの高い位置を取得し、
テンプレートと位置のタプルをlocsに加えてゆく。
この手順をすべてのテンプレートに対して実行する。
matchTemplateでマッチングを行い、
最大スコアをminMaxLocで取得するという一連の流れは、
OpenCVによるテンプレートマッチングの基本的な流れである。
minMaxLocの返り値は複雑であるが、ドキュメントに記載されている\cite{minMaxLoc}通り
Core.MinMaxLocResult型である。
プログラムでは座標を返すmaxLocフィールドと値を返すmaxValフィールドを参照する。

%疑問提起
\begin{comment}
val ((tile, loc), i) = locs.zipWithIndex.maxBy(_._1._2.mavVal)
について、
どうしてひとつのテンプレートの値だけに着目しているのか
他のテンプレートの値はどうなっている?
ループで呼ぶのか?
だとしたらval locs= { ... }
の処理が毎回呼ばれる分処理に問題がないか?
\end{comment}
%解答
\begin{comment}
goのループで読んでいる。
処理は大変重たい。
しかしminMaxLocの関数の仕様上、
matchTemplate -> minMaxLocという
openCVのマッチングの流れ上、
仕方のない話なのである。
\end{comment}

すべてのテンプレート(及びその回転)に対してのマッチングの結果locsから、
loc値が最大となるものをmaxByによって取得し、変数((tile,loc),i)に代入する。
スコアが最大となる位置はloc.maxLocで参照でき、
それにテンプレートの大きさtile.sizeを組み合わせることで、
マッチした手牌の輪郭rectを得られる。

ここでgoの引数を見てみる。
goの引数はrectsであり、
それはrectと、そのインデックス番号のタプルのシーケンスである。
goは再帰を前提としており、このシーケンスrectsは再帰のたびに認識結果を積み重ねてゆく。

rectを得たあと、引数のrectsについて、

%疑問
\begin{comment}
・43行目case(hand,contour) if hand.size.area>0の意味
・pair => !intersects ... の処理がわからない
・rectangle?矩形を描く?Imgproc.rectangleが見つからない...
\end{comment}


%\begin{lstlisting}[caption=,label=]
%
%\end{lstlisting}

\begin{thebibliography}{9}
\bibitem{yoshida}
	Sanshiro Yoshida.mahjongs.\\
	https://github.com/halcat0x15a/mahjongs
\bibitem{minMaxLoc}
	OpenCV 2.4.2 Java API.Core.MinMaxLocResult.\\
	http://docs.opencv.org/java/2.4.2/index.html?org/opencv/core/Core.html
\end{thebibliography}

\end{multicols}

\begin{comment}
\newpage

\section{付録}

\subsection{画像認識}
\begin{lstlisting}[caption=TemplateMatching.scala,label=TemplateMatching]
package mahjongs.recognizer

import org.opencv.core._
import org.opencv.imgproc.Imgproc

object TemplateMatching {

  val MaxResolution: Int = 1024 * 1024

  def createTemplate(mat: Mat): Option[(IndexedSeq[Mat], Size)] = {
    crop(resize(mat, MaxResolution)).headOption.map {
      case (template, _) =>
        val mask = threshold(template.clone, true)
        val width = template.cols / 9
        val height = template.rows / 4
        floodFill(mask, (0 until 4).map(_ * height) :+ (mask.rows - 1), 0 until mask.cols)
        floodFill(mask, 0 until mask.rows, (0 until 9).map(_ * width) :+ (mask.cols - 1))
        Imgproc.dilate(mask, mask, new Mat)
        val rects = for (tiles <- grid(mask, 4, 9)) yield {
          for (tile <- tiles) yield {
            val contours = findContours(tile.clone).flatMap(_.toArray)
            if (contours.length > 0)
              Some(convexHull(contours))
            else
              None
          }
        }
        val size = rects.flatten.flatten.map(_.size).maxBy(_.area)
        val tiles = grid(template, 4, 9).zip(rects).flatMap {
          case (tiles, rects) =>
            tiles.zip(rects).map {
              case (tile, rect) =>
                Imgproc.getRectSubPix(tile, size, rect.fold(center(tile))(center), tile)
                tile
            }
        }.take(34)
        (tiles, new Size(width, height))
    }
  }

  def recognize(mat: Mat, templates: Seq[Mat], width: Int, height: Int): (Seq[Int], Seq[Seq[Int]]) = {
    val result = crop(resize(mat, MaxResolution)).collect {
      case (hand, contour) if hand.size.area > 0 =>
        Imgproc.resize(hand, hand, new Size(hand.size.width * height / hand.size.height, height))
        val edge = approxPoly(new MatOfPoint2f(contour.toArray: _*)).rows
        val tiles = templates ++ templates.map(m => flip(m.t, 0)) ++ templates.map(flip(_, -1)) ++ templates.map(m => flip(m.t, 1))
        def go(rects: List[(Int, Rect)]): List[(Int, Rect)] = {
          val locs = for (tile <- tiles) yield {
            val result = new Mat
            Imgproc.matchTemplate(hand, tile, result, Imgproc.TM_CCORR_NORMED)
            (tile, Core.minMaxLoc(result))
          }
          val ((tile, loc), i) = locs.zipWithIndex.maxBy(_._1._2.maxVal)
          val rect = new Rect(loc.maxLoc, tile.size)
          if (rects.forall(pair => !intersects(rect, pair._2))) {
            Imgproc.rectangle(hand, rect.tl, rect.br, new Scalar(0), -1)
            go((i % 34, rect) :: rects)
          } else {
            rects
          }
        }
        val indices = go(Nil).sortBy(_._2.x).map(_._1)
        (edge == 4 && indices.size != 4, indices)
    }.groupBy(_._1).mapValues(_.map(_._2))
    (result(true)(0), result.get(false).toList.flatten)
  }

}
\end{lstlisting}

\subsection{ユーザ定義関数}
\begin{lstlisting}[caption=package.scala,label=package]
package mahjongs

import scala.collection.JavaConverters._
import scala.collection.mutable.Buffer

import org.opencv.core._
import org.opencv.imgcodecs.Imgcodecs
import org.opencv.imgproc.Imgproc

package object recognizer {

  for {
    ext <- sys.props("os.name").toLowerCase match {
      case name if name.contains("nix") => Some("so")
      case name if name.contains("mac") => Some("dylib")
      case _ => None
    }
  } System.load(getClass.getResource(s"/libopencv_java300.$ext").getPath)

  def findContours(mat: Mat): Buffer[MatOfPoint] = {
    val contours = Buffer.empty[MatOfPoint]
    Imgproc.findContours(mat, contours.asJava, new Mat, Imgproc.RETR_EXTERNAL, Imgproc.CHAIN_APPROX_TC89_KCOS)
    contours.sortBy(Imgproc.contourArea)(Ordering.Double.reverse)
  }

  def floodFill(mat: Mat, rows: Seq[Int], cols: Seq[Int]): Mat = {
    val mask = new Mat
    val color = new Scalar(0)
    for (row <- rows; col <- cols if mat.get(row, col)(0) > 0)
      Imgproc.floodFill(mat, mask, new Point(col, row), color)
    mat
  }

  def grid(mat: Mat, rows: Int, cols: Int): IndexedSeq[IndexedSeq[Mat]] = {
    val height = mat.rows / rows
    val width = mat.cols / cols
    for (row <- 0 until rows) yield {
      val rowRange = new Range(row * height, (row + 1) * height)
      for (col <- 0 until cols) yield
        mat.submat(rowRange, new Range(col * width, (col + 1) * width))
    }
  }

  def approxPoly(contour: MatOfPoint2f, epsilon: Double = 0.01): MatOfPoint2f = {
    val curve = new MatOfPoint2f
    Imgproc.approxPolyDP(contour, curve, Imgproc.arcLength(contour, true) * epsilon, true)
    if (curve.rows % 2 == 0)
      curve
    else
      approxPoly(contour, epsilon + 0.01)
  }

  def threshold(mat: Mat, inv: Boolean): Mat = {
    val (tpe, op) = if (inv) (Imgproc.THRESH_BINARY_INV, Imgproc.MORPH_OPEN) else (Imgproc.THRESH_BINARY, Imgproc.MORPH_CLOSE)
    Imgproc.threshold(mat, mat, 0, 255, tpe | Imgproc.THRESH_OTSU)
    Imgproc.morphologyEx(mat, mat, op, new Mat)
    mat
  }

  def crop(mat: Mat): Buffer[(Mat, MatOfPoint)] = {
    for (contour <- findContours(threshold(mat.clone, false))) yield {
      val patch = new Mat
      val rect = Imgproc.minAreaRect(new MatOfPoint2f(contour.toArray: _*))
      Imgproc.warpAffine(mat, patch, Imgproc.getRotationMatrix2D(rect.center, rect.angle, 1), mat.size)
      Imgproc.getRectSubPix(patch, rect.size, rect.center, patch)
      if (rect.angle <= -45) Core.flip(patch.t, patch, 0)
      (patch, contour)
    }
  }

  def convexHull(contours: Seq[Point]): Rect = {
    val hull = new MatOfInt
    Imgproc.convexHull(new MatOfPoint(contours: _*), hull, false)
    Imgproc.boundingRect(new MatOfPoint(hull.toArray.map(contours): _*))
  }

  def flip(mat: Mat, code: Int): Mat = {
    val m = new Mat
    Core.flip(mat, m, code)
    m
  }

  def resize(mat: Mat, max: Int): Mat = {
    val r = math.sqrt(mat.rows * mat.cols / max.toDouble)
    if (r > 1) Imgproc.resize(mat, mat, new Size(mat.size.width / r, mat.size.height / r))
    mat
  }

  def center(mat: Mat): Point =
    new Point(mat.size.width / 2, mat.size.height / 2)

  def center(rect: Rect): Point =
    new Point(rect.x + rect.width / 2, rect.y + rect.height / 2)

  def toMat(bytes: Array[Byte], gray: Boolean): Mat =
    Imgcodecs.imdecode(new MatOfByte(bytes: _*), if (gray) Imgcodecs.CV_LOAD_IMAGE_GRAYSCALE else Imgcodecs.CV_LOAD_IMAGE_COLOR)

  def fromMat(mat: Mat): Array[Byte] = {
    val buf = new MatOfByte
    Imgcodecs.imencode(".png", mat, buf)
    buf.toArray
  }

  def intersects(a: Rect, b: Rect): Boolean =
    math.max(a.x, b.x) < math.min(a.x + a.width, b.x + b.width) && math.max(a.y, b.y) < math.min(a.y + a.height, b.y + b.height)

  def read(filename: String, gray: Boolean): Mat =
    Imgcodecs.imread(filename, if (gray) Imgcodecs.IMREAD_GRAYSCALE else Imgcodecs.IMREAD_COLOR)

}
\end{lstlisting}
\end{comment}

\end{document}